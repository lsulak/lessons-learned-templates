\cvsection{Work Experience}
\begin{cventries}

  \cventry
    {Lead Data Engineer} 
    {Ecotricity} 
    {Bristol, England} 
    {June 2021 - January 2022}
    {
        \begin{cvitems} 
            \item {I led a small data engineering unit, with responsibilities ranging from technical ownership, product ownership, and team ownership. I was driving the whole software development life-cycle, which included consulting, delegating, or actively executing individual parts of the SDLC. Also, I facilitated all the Scrum meetings.}
        \end{cvitems}
 	}
  
  \cventry
    {Senior Data Engineer}
    {} % Organization EMPTY
    {} % Location EMPTY
    {October 2020 - June 2021} % Date(s)
    {
      \begin{cvitems}
        \item {Additional responsibilities involved coaching junior engineers, having greater ownership over internal and business projects, and communicating more with the key stakeholders.}
        \end{cvitems}
 	}
 	

  \cventry
    {Data Engineer}
    {} % Organization EMPTY
    {} % Location EMPTY
    {October 2019 - October 2020} % Date(s)
    {
      \begin{cvitems}
      	\item {Being an early member of a growing data engineering function that was continuously enabling data democratization and data literacy within the organization. Also, we developed many internal products requested by stakeholders from various parts of the company.}
      	\item {Our approach primarily involved Python and Spark-based data pipelines deployed to the AWS, but some of our resources were deployed on-premises.}
      	\item {As our team reached a more mature state, I developed many reusable components and previously not existing automation, which led to faster development and higher-quality products. I put a big emphasis on doing peer reviews, writing tests and documentation, applying best practices, implementing CI/CD and IaC, and sharing the lessons within the team and, if relevant, also with a broader audience in the company. Some of our products reached more than 90\% of the test coverage and we could easily have multiple deployments per day.}
    %   	\item {Thanks to a proper choice of technology (Databricks \& Lakehouse) and regular meetups, we were able to break down silos and unify analysts, engineers, and many line managers so that everyone was using the same platform and language. Since I was one of the first users of Databricks in our organization, I had a deep understanding of the platform and I was a very active part of the overall transformation process.}
      	\item {I continuously tried to identify gaps and more challenging technical problems in regard to my team's responsibilities and beyond. Many technologies we were using were relatively new and this sometimes required doing research, writing PoCs, and being innovative and curious in general.}
        \item {{\bodyfont\bfseries\color{darktext}Tech stack:} 
            \begin{itemize}
                \item {Python 3: PySpark, Pandas, Jupyter, Great Expectations, Boto3, Troposphere, Conda, Black, Pylint, Pytest, ...}
                \item {On-premise: MS-SQL and (Power/Bash)Shell}
                \item {Cloud: Databricks and AWS: S3, Glue, DMS, Athena, DynamoDB, Lambda, SageMaker, CloudFormation, ...}
                \item {Environment \& CI/CD: Linux/WSL, Docker, AzureDevOps, BitBucket and GitHub}
            \end{itemize}
      \end{cvitems}
 	    }
    }
    
 	
 \cventry
    {Software Developer}
    {ICT \& MEDIA, s.r.o.}
    {Brno, Czech Republic}
    {January 2019 - August 2019}
    {
      \begin{cvitems}
      	\item {Designed and developed a back-end of a threat intelligence platform. Responsibilities went from a low-level point of view, such as code quality, to a high-level point of view, such as choosing the right frameworks, software architecture, and active consultation of requirements with a remote project manager.}
        \item {Actively participating in technical interviews for bringing new people in. Wrote a practical assignment and interview test for a python developer position.}
        \item {Gave a lot of practical advice and put a great emphasis on the overall quality of the project.}
        \item {Developed a testing framework that resulted in much easier testing and thus higher coverage (more than 90\% overall).}
        \item {{\bodyfont\bfseries\color{darktext}Tech stack:} Python 3 (Celery, Pipenv, Black, Pylint, Pytest, Protobuf), PostgreSQL, SQLite, Linux, Docker, and GitLab.}
      \end{cvitems}
    }

 \cventry
    {Software Developer and Analyst}
    {CYAN Research \& Development s.r.o.}
    {Brno, Czech Republic}
    {July 2016 - December 2018}
    {
      \begin{cvitems}
      	\item {Participating in building back-end of a bigger system for detecting internet fraud, such as malicious web servers, and web-based phishing.}
   	   \item {The nature of the work was not just the design, development, and maintenance of software systems, but also domain expertise, and research.}
        \item {Helped the company with a lot of research ideas and practical applications, such as a module that stored a high amount of domains generated from domain generation algorithms hidden in malware. Because of this, the customers were more protected against C\&C attacks.}
        \item {Occasionally volunteered on having a tech talk about certain topics, and current technical problems so that they would not be repetitive.}
        \item {{\bodyfont\bfseries\color{darktext}Tech stack:} Python 3 (Scrapy, Numpy, Scikit-learn, Keras), Bash and various Linux tooling, PostgreSQL, Grafana, Ansible, Yara, and GitLab.}
      \end{cvitems}
    }
    
%   \cventry
%     {Software Developer}
%     {IXPERTA}
%     {Brno, Czech Republic}
%     {March 2015 - December 2015}
%     {
%       \begin{cvitems}
%         \item {Successful migration of a client-server application from .NET 2.0 to .NET 4.5. This was done by replacing obsolete libraries and other parts of the system. Dedicated to the client part migration, that visualized incoming data (memory leaks on remote machine).}
%         \item {Implementation of a module that replaced an older non-functional Windows USB driver. The final solution was more lightweight, maintainable, and reliable.}
%         \item {{\bodyfont\bfseries\color{darktext}Tech stack:} Mostly C\# on .NET platform (occasionally also C/C++), Libusb and various USB monitoring tools, Subversion and ClearCase.}
%       \end{cvitems}
%     }

\end{cventries}

\newpage
